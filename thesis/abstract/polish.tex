%!TEX output_directory = texaux
%!TEX spellcheck
%!TEX root = ../main.tex

Praca stanowi przedstawienie niektórych mechanizmów neuronowych wykorzystywanych przy tworzeniu systemów dialogowych, czyli programów komputerowych potrafiących prowadzić rozmowę z użytkownikiem. Pokazuję również jak opisywane metody sprawdzają się w praktyce.

Zaimplementowałem i przetestowałem algorytmy do statystycznego modelowania oraz generowania dialogu. Eksperymentowałem także z architekturą odpowiadającą na pytania o fakty zadawane w języku naturalnym. Do tej ostatniej części dołączam moją próbę rozszerzenia możliwości algorytmu.

