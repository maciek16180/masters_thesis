%!TEX output_directory = texaux
%!TEX spellcheck
%!TEX root = ../main.tex

\setlength{\abovedisplayskip}{20pt}
\setlength{\belowdisplayskip}{20pt}

\chapter{Podsumowanie}

Sieci neuronowe znalazły w ostatnich latach szerokie zastosowanie w dziedzinie przetwarzania języka, na czym naturalnie skorzystały systemy dialogowe. Generowanie zdań słowo po słowie jest ciekawą perspektywą, która uwalnia nas od konieczności ręcznej konstrukcji wypowiedzi i daje maszynie możliwość wykazania oryginalności. Niestety obecnie to rozwiązanie boryka się z dość fundamentalnymi przeszkodami. Różnorodność wypowiedzi i dobre zrozumienie kontekstu są bardzo trudne do uzyskania. Automatyczna ocena jakości generowanych dialogów dobrze korelująca z osądem ludzkim cały czas pozostaje otwartym problemem. Niemniej jednak jest to interesująca gałąź sztucznej inteligencji, ciesząca się zainteresowaniem w środowiskach naukowych.

Wyszukiwanie odpowiedzi na pytania w zadanym kontekście daje znacznie bardziej imponujące wyniki. Najlepsze systemy osiągają w tym zadaniu skuteczność bardzo bliską ludzkiej, co pozwala z powodzeniem wykorzystywać je w praktyce. Sporą trudnością pozostaje jednak znalezienie odpowiedniego tekstu do konkretnego pytania. Wykorzystanie standardowych technik przeszukiwania nie zawsze daje pożądane rezultaty. Optymalny system powinien być wyposażony w wyszukiwarkę specjalnie dostosowaną do tego problemu lub potrafić samemu ocenić dopasowanie otrzymywanych fragmentów.

Niemałą rolę w rozwoju systemów dialogowych odgrywa rywalizacja. Konkursy takie jak The Alexa Prize\footnote{\url{https://developer.amazon.com/alexaprize}}, Nagroda Loebnera, czy NIPS Conversational Intelligence Challenge stanowią coroczną motywację dla naukowców i programistów. Łatwa dostępność chatbotów, na przykład za pośrednictwem Facebooka, sprawia, że są one coraz mocniej obecne w świadomości publicznej. Mnogość interaktywnych aplikacji asystujących użytkownikowi pokazuje, że roboty konwersacyjne mają również potencjał w biznesie. Systemy dialogowe stają się więc coraz bardziej popularne i~zaawansowane.